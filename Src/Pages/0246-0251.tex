

    \begin{zad}

        Skup \dj aka iz odeljenja F = \{Petar, Ana, Vera, Igor, Jovan \} izjavili su da su videli neke od gradova koji \v cine skup G = \{Beograd, Ni\v s, Novi Sad, Subotica \}. Beograd su videli Petar i Vera, Ni\v s su videli Vera i Jovan, Novi Sad je video Igor. Prika\v zi tu relaciju.

    \end{zad}



    \begin{figure}[H] 

        \center

\input{ Slike/0246-0251-slika1 }
        \caption{Slika 200}

        \label{Slika200}

    \end{figure}



    Prikazana je relacija ,,Video je''.



    Skup F, iz koga izlaze strelice zove se izvor, a skup G zove se cilj.



    Ovo je prvi na\v cin prikazivanja relacije pomo\' cu strelica (sagitalna \v sema).



    Ista relacija se mo\v ze prikazati i ovako:



    \begin{figure}[H] 

        \center

\input{ Slike/0246-0251-slika2 }
        \caption{Slika 201}

        \label{Slika201}

    \end{figure}



    Ovaj drugi na\v cin prikazivanja relacije zove se mre\v za ili Dekartova \v sema.



    \begin{zad}

        Skup \dj aka odeljenja D = \{Nikola, Jovan, Vera, Igor, Marija \} ima na \v stednji neki od \v stednih uloga skupa S = \{10000, 20000, 40000, 100000 \}.



        Zna se da po 40000 imaju Nikola i Igor, 10000 ima Marija, a Vera i Jovan imaju po 20000.



        Prikazati na oba na\v cina ovu relaciju: ,,ima na \v stednju''.

    \end{zad}



    \begin{zad}

        Slika \ref{Slika202} prikazuje ,,ima za brata''. Na koji na\v cin je prikazana ova relacija? Koji su elementi mu\v ski, a koji \v zenski? Prikazati ovu relaciju na drugi na\v cin.

    \end{zad}



    \begin{figure}[H] 

        \center

\input{ Slike/0246-0251-slika3 }
        \caption{Slika 202}

        \label{Slika202}

    \end{figure}



    Relacija je prikazana pomo\' cu strelice (sagitalna \v sema) i to u istom skupu.



    Ako a ,,ima za brata'' b i b ,,ima za brata'' a, onda su a i b mu\v ski elementi. Strelice dalje kazuju da d ,,ima za brata'' c, e ,,ima za brata'' c, i h ,,ima za brata'' g, \v sto zna\v ci da su c i g mu\v ski elementi, a \v zenski su d, e i h.



    Drugi na\v cin prikazivanja ove relacije je pomo\' cu njene mre\v ze  ili Dekartove \v seme.



    Obrati pa\v znju na prikazivanje Dekartove \v seme, jer je izvor i cilj isti skup.



    \begin{figure}[H] 

        \center

\Placeholder[(3, 3)]{Ovde stoji opis ovog crteza}
        \caption{Slika 203}

        \label{Slika203}

    \end{figure}



    \begin{zad}

        Na Slici \ref{Slika204} Dekartova \v sema prikazuje relaciju ,,je manji od''. Prikazati relaciju pomo\' cu strelice (sagitalne \v seme).

    \end{zad}



    \begin{figure}[H] 

        \center

\input{ Slike/0246-0251-slika5 }
        \caption{Slika 204}

        \label{Slika204}

    \end{figure}



    \v Sema relacije ,,je manji od'' pomo\' cu strelica (sagitalne \v seme) prikazana je na Slici \ref{Sliika205}.



    \begin{figure}[H] 

        \center

\input{ Slike/0246-0251-slika6 }
        \caption{Slika 205}

        \label{Slika205}

    \end{figure}



    \begin{zad}

        Na Slici \ref{Slika206} prikazana je relacija ,,deli'' pomo\' cu strelica (sagitalna \v sema).

    \end{zad}



    \begin{figure}[H] 

        \center

\input{ Slike/0246-0251-slika7 }
        \caption{Slika 206}

        \label{Slika206}

    \end{figure}



    Obrati pa\v znju da relacija ,,deli'' zna\v ci da pri deljenju broja u koji ulazi strelica brojem iz kog izlazi strelica, ostatak je 0.



    Svaki broj je deljiv samim sobom, a strelica koja polazi iz jednog elementa i vra\' ca se u isti element pokazuje da 1 deli 1, 2 deli 2, 3 deli 3, i zove se alka (petlja).



    Nacrtaj Dekartovu \v semu (mre\v zu) ove relacije.



    \begin{figure}[H] 

        \center

\input{ Slike/0246-0251-slika8 }
        \caption{Slika 207}

        \label{Slika207}

    \end{figure}



    \begin{zad}

        U skupu M = \{10,3,2,5,6,15,18\} prikazati obe \v seme relacije ,,je ve\' ci od'' i relacije ,,deli''.

    \end{zad}



    \begin{zad}

        U skupu A = \{Petar, Persa, Proha \}, B = \{Petar, Persa, Milan \}, C = \{Petar, Vera, Milan \} prikazati ove \v seme relacije: ,,njegovo ime po\v cinje istim slovom kao i moje''.



        Posmatrajmo skup A i neka svaki element ovog skupa poka\v ze element \v cije ime po\v cinje slovom kao njegovo ime.

    \end{zad}



    Neka je Petar element x, Persa element y, i Proha element z. Onda x pokazuje y i izgovara: ,,njegovo ime...''. Ova dva iskaza se prikazuju suprotnim strelicama. Isti iskazi se prikazuju i izgovaraju x i z, y i z.



    Sagitalna \v sema izgleda ovako:



    \begin{figure}[H] 

        \center

\input{ Slike/0246-0251-slika9 }
        \caption{Shemabezimena}

        \label{Shemabezimena}

    \end{figure}



    \v Sema nije gotova. Za\v sto?



    Svaki element pokazuje ostala dva elementa i izgovara ,,njegovo ime...'', ali je zaboravio da poka\v ze sebe i izgovori i ,,moje ime po\v cinje slovom kao moje ime (slovo)''.



    Za\v sto kona\v cna (gotova) \v sema izgleda ovako:



    \begin{figure}[H] 

        \center

\input{ Slike/0246-0251-slika10 }
        \caption{Slika 208}

        \label{Slika208}

    \end{figure}



    B = \{Petar, Persa, Milan \}, gde je Petar x, Persa y i Milan m.



    \begin{figure}[H] 

        \center

\input{ Slike/0246-0251-slika11 }
        \caption{Slika 209}

        \label{Slika209}

    \end{figure}



    C = \{Petar, Sara, Milan \}, gde je Petar x, Sara s, Milan m.



    \begin{figure}[H] 

        \center

\input{ Slike/0246-0251-slika12 }
        \caption{Slika 210}

        \label{Slika210}

    \end{figure}



    \begin{zad}

        U skupu D = \{Ana, Aleksa, Aca, Andra \} prikazati obe \v seme relacije: ,,njegovo ime po\v cinje istim slovom kao moje''.

    \end{zad}



    D = \{Ana, Aleksa, Aca, Andra \}. Neka je Ana x, Aleksa y, Aca z, Andra m, tj D = \{x,y,z,m \}.



    \begin{figure}[H] 

        \center

\input{ Slike/0246-0251-slika13 }
        \caption{Slika 211}

        \label{Slika211}

    \end{figure}

    

