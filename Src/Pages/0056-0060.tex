    \LaTeX

    \begin{zad}

        Sastavi razliku skupova \{a, b, c, d\} i \{b, c, d, f, g, h\}

        \

    \end{zad}

        

    \begin{zad}

        Sastavi skup \v ciji su su elementi: blok u kome crta\v s,

        olovka kojom pi\v se\v s, guma kojom bri\v se\v s, reza\v c kojim o\v stri\v s.

        \

        

        Treba zagraditi ove elemente kanapom i tim obrazuje\v s skup \{b,o,g,z\} 

        \ \ skloni sve te elemente. Sklonjeni su svi elementi datog skupa. \v Sta je ostalo?

        \

        

        Ostao je prazan skup.

        Zapi\v si to

        \ 

        

        $\{b, o, q, z\} \setminus \{b, o, q, z\} = \{ \ \} =

        \emptyset$

        \ 

    \end{zad}

    \begin{zad}

        Sastavi skup malih kru\v znih \v zetona. Koja je razlika izme\dj u tog skupa i jednakog skupa, tj istog skupa? Zapi\v si to.

        \ 

        

        \{ kru\v zni mali  \v zetoni \} $\setminus$ \{ mali kru\v zni \v zetoni \} = \{  \}

        \ 

    \end{zad}

    \begin{zad}

        U\v cenici koji igraju fudbal za vreme \v skolskog odmora na \v skolskom igrali\v stu \v cine skup. Posle \v skolskog zvona za po\v cetak \v casa, u\v cenici su oti\v sli na \v cas. Na \v skolskom igrali\v stu ostao je prazan skup.

        \

        

        Mogu da zaklju\v cim, kada je razlika dva skupa prazan skup.

        \ 

        Kada?

        \

        

        Razlika dva jednaka (\ tj. \ ista )\ skupa je prazan skup.

        \

    \end{zad}

    \begin{zad} 

        Sastavi skup plavih \v zetona. Koja je razlika izme\dj u tog skupa i istog skupa  tj. jednakog skupa? Zapi\v si.

        \

    \end{zad}

    \begin{zad} 

        Sastavi skup pravougaonih crvenih \v zetona i skup kru\v znih crvenih \v zetona.

        \

        Napravi razliku skupa pravougaonih crvenih \v zetona i skupa kru\v znih \v zetona. Objasni postupak.

        \ 

        

        Prvo odre\dj ujem da li skupovi imaju zajedni\v cke elemente. 

        

        Presek je prazan skup. Nema zajedni\v ckih elemenata. Skupovi su razdvojeni. 

        

        Zna\v ci, razlika ne odvaja nijedan element skupa pravougaonih crvenih \v zetona, tj. skup pravougaonih crvenih \v zetona ostaje netaknut i zato je razlika: skup pravougaonih \v zetona. 

        \ 

        

        Prema tome: 

        \ 

        

        \{pravougaoni  crveni \v zetoni\}$\setminus$\{kru\v zni  crveni  \v zetoni\} = \

        \{pravougaoni crveni \v zetoni\}

        \

        \ 

        Utvrdi razliku skupa kru\v znih crvenih \v zetona i skupa pravougaonih crvenih \v zetona.

        Obrazlo\v zi postupak. 

        \  

        

        Koristim prethodni postupak utvr\dj ujem da je: 

        \ 

        

        \{kru\v zni crveni \v zetoni \} $\setminus$ \{pravougaoni crveni \v zetoni\} = \

        \{kru\v zni crveni \v zetoni\}

    \end{zad}

    \begin{zad} 

        Napi\v si razliku skupova \{sveska, olovka, \v sestar \} i \{guma, knjiga, kesten, banana\} 

        \ 

        

          Dati skupovi nemaju zajedni\v ckih elemenata tj. presek je prazan skup, zato je: \

          

          \{sveska, olovka, \v sestar\} $\setminus$ \{guma, knjiga, kesten, banana\} = \ \{sveska, olovka, \v sestar \}

        \  

        

        Uo\v cavas da je razlika dva data skupa prvi skup. Za\v sto? 

        \ 

        Zato \v sto dati skupovi nemaju zajedni\v ckih elemenata.

        \ 

        \ 

        Zatim, Razlika dva data skupa koji nemaju zajedni\v ckih 

        \ 

        Elemenata je prvi skup.

        \

    \end{zad}

    \begin{zad}

    Odredi presek skupova i prika\v zi Venovim dijagramima: 

    \ 

    

    1)\ A = \{1,2,3,4,5\} i B = \{4,5,6,7,8\}, C = \{1,2,3,4\} i D = \{6,7,8,9,10\}.

    



    2)\ A $\cap$ B = \{1,2,3,4,5\} $\cap$ \{4,5,6,7,8\} = \{4,5\}



    

    \begin{figure}[h] 

        \center

\input{ Slike/0056-0060-slika1 }
        \caption{}

    \end{figure}



    Ako dva skupa imaju zajedni\v cke elemente, oni (\ ti zajedni\v cki elementi )\ \v cine skup koji se zove presek datih skupova.

    \

    \newpage

    2)\ C $\cap$ D = \{1,2,3,4\} $\cap$ \{6,7,8,9,10\} = \{ \ \}

    

    \begin{figure}[h] 

        \center

\input{ Slike/0056-0060-slika2 }
        \caption{}

    \end{figure}

    

    Ako dva skupa nemaju zajedni\v cke elemente. Njihov presek je prazan skup.

\ 

\end{zad}

\begin{zad}

    Odredi uniju skupova A = \{1,2,3,4,5\} i B = \{4,5,6,7,8\}. 

    \ 

    Prika\v zi uniju Venovim Dijagramima. 

    \ 

    

    Odre\dj ujem A $\cap$ B (\ zajedni\v cke elemente skupova A i B)\ . 



    A $\cap$ B = \{4,5\} (\ vidi zadatak 136.1)

    

    \begin{figure}[h] 

        \center

\input{ Slike/0056-0060-slika3 }
        \caption{}

    \end{figure}



    A $\cup$ B = \{1,2,3,4,5\} $\cup$ \{4,5,6,7,8\} = \{1,2,3,4,5,6,7,8\}

    \ 

    

    Unija skupova A i B je skup \v ciji su elementi skupa A ili skupa B ( \ obrati pa\v znju na veznik ili: To su elementi skupa  A ili skupa B. U vezi sa veznikom ili vidi zadatak 113. ) \

    \ 



    Uo\v ci da su zajedni\v cki elementi u\v sli kao elementi skupa A ( \ prvog skupa ) \, i oni ne ulaze kao elementi drugog skupa B. 

    \ 

    Odredi uniju skupova A = \{1,2,3,4\} i \{6,7,8\}. 

    \ 

    

    A $\cap$ B = \{1,2,3,4\} $\cap$ \{6,7,8\} = \{ \ \} 



 \begin{figure}[h] 

        \center

\input{ Slike/0056-0060-slika4 }
        \caption{}

\end{figure}





A $\cup$ B = \{1,2,3,4\} $\cup$ \{6,7,8\} = \{1,2,3,4,5,6,7,8\}

\

\end{zad}

\begin{zad}

    Odredi razliku skupova A = \{1,2,3,4,5,6\} i B = \{4,5,6\}.

    \ 





    A $\cap$ B = \{1,2,3,4,5,6\} $\cap$ \{4,5,6\} = \{4,5,6\}.

\ 

\

    Skup B je podskup ( \ deo ) \ skupa A. 

\ 

\ 





    A $\setminus$ B = \{1,2,3,4,5,6\} $\setminus$ \{4,5,6\} = \{1,2,3\}

 \ 



  \begin{figure}[h] 

        \center

\input{ Slike/0056-0060-slika5 }
        \caption{}

\end{figure}



    Razlika skupa A i njegovog podskupa B je dopunski skup skupa B u odnosu na skup A.

    \ 



    Odredi razliku skupa A = \{1,2,3,4,5,6\} i skupa C = \{1,2,3\}. 

    \ 

    

    A $\cap$ C = \{1,2,3,4,5,6\} $\cap$ \{1,2,3\} = \{4,5,6\}

    \ 

    

    Razlika skupa A i njegovog podskupa C je dopunski skup skupa C u odnosu na skup A

\end{zad}

\begin{zad}

    Odredi razliku skupova A = \{1,2,3,4,5\} i B = \{3,4,5,6,7,8\}. 

    \ 

    Prikazati Venovim dijagramima. 

    \ 



     A $\cap$ B = \{1,2,3,4,5\} $\cap$ \{3,4,5,6,7,8\} = \{3,4,5\}.

     \ 



     \begin{figure}[h] 

        \center

\Placeholder[(4, 1)]{Ovde stoji opis ovog crteza}
        \caption{}

\end{figure}



    Skupovi imaju neke zajedni\v cke elemente, onda moram da vodim ra\v cuna koju razliku ho\' cu da odredim.

    \ 



    A $\setminus$ B = \{1,2,3,4,5\} $\setminus$ \{3,4,5,6,7,8\} = \{1,2\} 

    \ 

    

    B $\setminus$ A = \{3,4,5,6,7,8\} $\setminus$ \{1,2,3,4,5\} = \{6,7,8\}

    \ 

\end{zad}

\begin{zad}

    Odredi razliku skupova A = \{1,2,3,4\} i B = \{6,7,8\}. 

    \ 

    Prikazi Venovim dijagramima. 

    \ 



    A $\cap$ B = \{1,2,3,4\} $\cap$ \{6,7,8\} = \{ \ \} 



    Presek je prazan skup. 



\begin{figure}[h] 

        \center

\input{ Slike/0056-0060-slika7 }
        \caption{}

\end{figure}

    



    A $\setminus$ B = \{1,2,3,4\} $\setminus$ \{6,7,8\} = \{1,2,3,4\} 



    B $\setminus$ A = \{6,7,8\} $\setminus$ \{1,2,3,4\} = \{6,7,8\} 

    \ 



    Dati skupovi nemaju zajedni\v ckih elemenata, onda je razlika prvi skup.



\end{zad}

\begin{zad}

    Odredi razliku skupova A = \{1,2,3,4\}, B = \{4,3,2,1\}

    \ 



    A $\cap$ B = \{1,2,3,4\} $\cap$ \{4,3,2,1\} = \{1,2,3,4\}

    \ 



    A = B su jednaki skupovi (\ razlikuju se u redosledu elemenata, koji nije bitan za skup )\ 

    \ 



    A $\setminus$ B = \{1,2,3,4\} $\setminus$ \{4,3,2,1\} = \{ \ \} 

    \ 



    B $\setminus$ A = \{4,3,2,1\} $\setminus$ \{1,2,3,4\} = \{ \ \}

    \ 

    \ 

    Razlika dva jednaka ( \ ista ) \ skupa je prazan skup.

    \

\end{zad}

\begin{zad}

    Odredi presek, uniju i razliku skupova\

    

    A = \{a,b,c,d,e\} i B = \{d,e,f,g,h\}, Prika\v zi Venovih dijagramima. 

    \ 

    

    A $\cap$ B = \{a,b,c,d,e\} $\cap$ \{d,e,f,g,h\} = \{d,e\}



    A $\cup$ B = \{a,b,c,d,e\} $\cup$ \{d,e,f,g,h\} = \{a,b,c,d,e,f,g,h\}



    A $\setminus$ B = \{a,b,c,d,e\} $\setminus$ \{d,e,f,g,h\} = \{a,b,c\}



    B $\setminus$ A = \{d,e,f,g,h\} $\setminus$ \{a,b,c,d,e\} = \{f,g,h\} \

    

    Dobijena su \v cetiri skupa:\



    1. A $\cap$ B = \{d,e\} zove se presek skupova A i B, u sastav preseka ulaze samo zajedni\v cki elementi datih skupova i kratka (\ simboli\v cki )\ se ozna\v cava: \ 



    A $\cap$ B [\ \v citaj: A $\cap$ B ]\



    2. A $\cup$ B = \{a,b,c,d,e,f,g,h\} zove se unija skupova A i B, u sastav unije ulazi svaki element skupa A i svaki element skupa B koji nije ve\v c u\v sao kao zajedni\v cki element ( \ preseka A $\cap$ B ) \ , ili obrnuto ( \ svaki element skupa B koji nije ve\v c u\v sao kao zajedni\v cki element ( \ preseka B $\cap$ A ) \ ) \ . Ozna\v cava se \ 



    A $\cup$ B [\ \v citaj : A unija B ]\

\end{zad}

