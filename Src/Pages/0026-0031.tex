

    \begin{enumerate}           %nabrajanje krece od prethode strane

    \item Pogledaj slede\' ci crte\v z sa nacrtanim'' skupovima slika \ref{slika1str26} s tim da vezivanjem'' jednog elementa skupa $ s jednim elementom skupa $ utvrdi\v s da li su skupovi ekvipotentni.

    

       \begin{figure}[h]

          \center

\Placeholder[(7, 4)]{Ovde stoji opis ovog crteza}
         \caption{Ovo je figura}\label{slika1str26}

       \end{figure} 

    

    Skupovi $ i $ su ekvipotentni: ima onoliko bokala koliko i stolova.



    Uo\v cava\v s da je crtanje ovih skupova (prethodne dve slike) dosta te\v zak posao i oduzima dosta vremena.



    Zato je \v covek tra\v zio i na\v sao na\v cin za lak\v se crtanje skupova. On zamenjuje (apstrahuje) sve ono \v sto nije va\v zno (bitno) za prikazivanje skupova (na primer boja, obim stabla, oblik bokala, boja cveta). Odlu\v cio se za ta\v cke koje ozna\v cavaju ma koje elemente.



    Uop\v ste, skupovi ta\v caka su izvanredno sredstvo za osposobljavanje \dj aka (u\v cenika) u apstrahovanju. A kako se i lako crtaju one su i najpreporu\v cljivije (najprakti\v cije) za prikazivanje skupova slika \ref{slika2str26}

        \begin{figure}[h]

            \center

\Placeholder[(7, 4)]{Ovde stoji opis ovog crteza}
            \caption{Ovo je figura}\label{slika2str26}

        \end{figure} 

    \item Skup tanjira $  i skup ka\v sika $ su ekvipotentni. Koristi ta\v cke za prikazivanje ovih skupova.

    \end{enumerate}

    

    \begin{zad} 

    Zamisli da svi u\v cenici tvog odeljenja gledaju TV emisiju. Ispitaj skup stolica u sali sa skupom koji \v cinite vi prisutni \dj aci. \v Sta se tada mo\v ze dogoditi?

       

    \end{zad}

       \begin{enumerate}

           \item Svaki \dj ak ima svoju stolicu i svaka stolica ima svoga \dj aka.



            \v Sto se Venovim dijagramom prikazuje slika \ref{slika1str27}

                   \begin{figure}[h]

                       \center

\Placeholder[(7, 4)]{Ovde stoji opis ovog crteza}
                       \caption{Ovo je figura}\label{slika1str27}

                    \end{figure} 

            

            

            Skup \dj aka $ i skup stolica $ su ekvipotentni: ima onoliko stolica koliko ima i \dj aka.



           \item Svaki \dj ak ima svoju stolicu, a svaka stolica nema svoga \dj aka.



            \v Sto se Venovim dijagramom prikazuje slika \ref{slika2str27}

                    \begin{figure}[h]

                       \center

\Placeholder[(7, 4)]{Ovde stoji opis ovog crteza}
                       \caption{Ovo je figura}\label{slika2str27}

                    \end{figure} 



            Skup \dj aka $ i skup stolica $ nisu ekvipotentni:



            \DJ aka ima manje nego stolica; ili stolica ima vi\v se nego \dj aka (ima stolica na kojima ne sede \dj aci).



            Ovo mo\v ze\v s re\' ci i ovako: Svakom elementu skupa \dj aka odgovara (pridru\v zuje) element skupa stolica, a svakom elementu skupa stolica ne odgovara element skupa \dj aka.



            \item Svaki \dj ak nema svoju stolicu, a svaka stolica ima svoga \dj aka;



            \v Sto se Venovim dijagramom prikazuje slika \ref{slika3str27}

                    \begin{figure}[h]

                       \center

\Placeholder[(7, 4)]{Ovde stoji opis ovog crteza}
                       \caption{Ovo je figura}\label{slika3str27}

                    \end{figure}



            Skup \dj aka $ i skup stolica $ nisu ekvipotentni; \dj aka ima vi\v se nego stolica ili stolica ima vi\v se nego \dj aka.



            Sada mo\v ze\v s re\' ci ovako: Svakom elementu skupa \dj aka ne odgovara element skupa stolica, a svakom elementu skupa stolica odgovara element skupa \dj aka.

       \end{enumerate}

    \begin{zad}

        Igore sastavi skup \v ciji su elementi razni znaci ekvipotentni skupu koji zove\v s: moja porodica; porodica moga druga Petra.

    \end{zad}

    \begin{zad}

        Skup tvojih drugova je ekvipotentan skupu ostavljenih'' olovaka na stolu, a skup ostavljenih'' olovaka je ekvipotentan skupu tvojih drugarica. \v Sta mo\v ze\v s re\' ci za skup tvojig drugova i tvojih drugarica?



        

        

    \end{zad}

       Ja imam onoliko drugova koliko je olovaka na stolu, a olovaka na stolu je onoliko koliko je mojih drugarica.\

       Prema tome: Ja imam onoliko drugova koliko i drugarica.



       To se mo\v ze re\' ci i kra\' ce: Skup mojih drugova je ekvipotentan sa skupom mojih drugarica.

    \begin{zad}

        Nikola ka\v ze: Ja imam onoliko klikera koliko Ivan.'', Ivan ka\v ze: Ja imam onoliko klikera koliko ima Jovan.''. \v Sta sad \v zna\v s?

    \end{zad}

    \begin{zad}

        Petar ka\v ze: Ja imam onoliko svezaka koliko Igor.'', Milan ka\v ze: Ja imam onoliko svezaka koliko Petar.''.  \v Sta sad zna\v s?

    \end{zad}

    \begin{zad}

        Ako je skup prisutnih \dj aka ekvipotentan skupu prisutnih'' stolica. \v Sta mo\v ze\v s re\' ci za skup prisutnih'' stolica?

    \end{zad}

    \begin{zad}

        \begin{enumerate}

            \item \v Sta mo\v ze\v s re\' ci za skup tvojih majki i skup Petrovih majki?

            \item \v Sta mo\v ze\v s re\' ci za skupove majki koje imaju pojedini ovde prisutni?

            \item \v Sta mo\v ze\v s re\' ci za skupove koji se sastoje od po jednog elementa?

        \end{enumerate}

    \end{zad}

    \begin{zad}

        Zamisli skup stolica u tvojoj u\v cionici. Da li je on ekvivalentan samom sebi?

    \end{zad}

    \begin{zad}

        \begin{enumerate}

            \item Zamisli skup prasadi koje vodi jedna krma\v ca. Da li je taj skup ekvipotentan samom sebi?(Dokle \' ce ta ekvipotentnost trajati?)

            \item Da  li je skup tvoje porodice ekvipotentan samom sebi?

            \item Da li je svaki skup ekvipotentan samom sebi?

        \end{enumerate}

    \end{zad}

    Ovim utvr\dj ujem da je svaki skup ekvipotentan samom sebi.



    Sada je trenutak da shvati\v s da je svaki element jednak'' samom sebi i dva predmeta ne mogu biti ista, dve \v skolske krede kupljene ovog trenutka u prodavnici, jo\v s neupotrebljavane nisu iste(vrati se na zadatak 20. i 21.)

    \begin{zad}

        Dragana ka\v ze: Ja imam manje bombona nego Milica.'', Milica ka\v ze:Ja imam manje bombona nego Jovana.''. \v Sta mo\v ze\v s re\' ci za skupove koje imaju Dragana i Jovana?

    \end{zad}

    \begin{zad}

        Petar je stariji od Nikole, a Nikola je mla\dj i od Jovana, koji je mla\dj i od Petra. Koji je od njih najmla\dj i? (I ovde je re\v c o skupovima, o skupovima godina).

    \end{zad}

    \begin{zad}

        Marko je mla\dj i od Gorana, a Goran je mla\dj i od Stevana. Koji je od njih najstariji?

    \end{zad}

    \begin{zad}

        Tetka Zora je odgajila vi\v se prasi\' ca nego tetka Mira, a baba Desa je odgajila vi\v se jari\' ca nego \v sto je tetka Zora odgajila prasi\' ca. \v Sta sad mo\v ze\v s re\' ci?

    \end{zad}

    \section{Formiranje pojma broj}

    \begin{zad}

        Posmatraj slede\' ce skupove prikazane Venovim dijagramom. \v Sta mo\v ze\v s re\' ci za slede\' ce skupove?

    \end{zad}

       \begin{figure}[h]

          \center

\Placeholder[(10, 3)]{Ovde stoji opis ovog crteza}
         \caption{Ovo je figura}

        \end{figure} 

     Skupovi $, $, $ i $ su ekvipotentni.



     

     Poka\v zi.



     Prvo \' cu pokazati vezivanjem elementa skupa $ sa elementom skupa $ (strelicom). Time pokazujem da su skupovi $ i $ ekvipotentni.

      \begin{figure}[h]

          \center

\Placeholder[(10, 3)]{Ovde stoji opis ovog crteza}
         \caption{Ovo je figura}\label{slika1str29}

     \end{figure} 



     Ovim je utvr\dj ujemo da su skupovi $ i $ ekvipotentni: Skup $ ima onoliko elemenata koliko i skup $.



     Koristim prethodni postupak i utvr\dj ujem da su $ i $ ekvipotentni skupovi: Skup $ ima onoliko elemenata koliko i skup $.



     Na osnovu prethodnog zaklju\v cujem:



     Ako su $ i $ ekvipotentni skupovi, $ i $ ekvipotentni skupovi, onda su i $ i $ ekvipotentni: Skup $ ima onoliko elemenata koliko i $. Zatim utvr\dj ujem da su $ i $ ekvipotentni skupovi, to zna\v ci i da su $ i $ ekvipotentni skupovi.



     Svi ovi skupovi na slici \ref{slika1str29}, su ekvipotentni i ta se ekvipotentnost, ta zajedni\v cka osobina svih tih skupova zove tri.



     Zato \' ce\v s ma koji skup koji je ekvipotentan sa skupovima $, $, $ i $ predstaviti skupom \v cije \' ce\v s elemente prikazati ta\v ckama ovako: 

        \begin{figure}[h]

            \center

\Placeholder[(4, 3)]{Ovde stoji opis ovog crteza}
            \caption{Ovo je figura}

        \end{figure} 



        Ovaj skup ima zajedni\v cku osobinu sa skupovima $, $, $ i $ i jo\v s sa mnogo skupova, koji se zovu tri.

        \begin{zad}

            A sada posmatraj skup (slika \ref{slika1str30}). Ima li on osobinu tri?

        \end{zad}

         \begin{figure}[h]

          \center

\Placeholder[(4, 3)]{Ovde stoji opis ovog crteza}
          \caption{Ovo je figura}\label{slika1str30}

        \end{figure} 



    Ne. On ima osobinu koja se zove \v cetiri.



   Sastavi bar jo\v s jedan skup \v cija se osobina zove \v cetiri.



   Na primer:

   \{sapun, olovka, \v ceki\' c, guma\}



   \{moj drug neboj\v sa, njegov pas, moja maca, moja kapa\}\

   Navedeni skupovi su ekvipotentni sa skupom (slika \ref{slika1str30}). Zajedni\v cka osobina tih skupova se zove \v cetiri.

   \begin{zad}

       Prika\v zi crte\v zom skup lopti kojima se sada igra\v s. Navedi jo\v s neki ekvipotentni skup. Kako se zove zajedni\v cka osobina tih skupova?

   \end{zad}

        \begin{figure}[h]

          \center

\Placeholder[(4, 3)]{Ovde stoji opis ovog crteza}
         \caption{Ovo je figura}\label{slika2str30}

        \end{figure} 



        Ekvipotentni skupovi sa skupom (slika \ref{slika2str30}) su na primer:



        

        \{sunce\}, \{mesec\}, \{direktor moje \v skole\}, \{moja majka\}, \ldots





        Zajedni\v cka osobina svih tih skupova se zove jedan.

        \begin{zad}

            Prika\v zi skup ekvipotentan skupu tvojih roditelja. Imenuj jo\v s neke ekvipotentne skupove. Kako se zove jo\v s jedna zajedni\v cka osobina tih skupova i jo\v s mnogih drugih ekvipotentnih skupova.

        \end{zad}

              \begin{figure}[h]

          \center

\Placeholder[(4, 3)]{Ovde stoji opis ovog crteza}
         \caption{Ovo je figura}\label{slika1str31}

        \end{figure} 



        Ekvipotentni skupovi sa skupom (slika \ref{slika1str31}) su na primer:



        

         \{moje noge\}, \{krila ptice\}, \{tabla, sun\dj er\}.



        Zajedni\v cka osobina svih tih skupova zove se dva.

   \begin{zad}

       Znam da mo\v ze\v s sastaviti mnogo skupova ekvipotentnih skupu prstiju tvoje ruke. Kako se zove zajedni\v cka osobina svih tih skupova i jo\v s mnogo drugih ekvipotentnih skupova?



       Zajedni\v cka osobina svih tih skupova zove se pet.

   \end{zad}

     

