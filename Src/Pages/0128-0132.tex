    \section{Mentalno zdravlje}

    \begin{zad}

        \begin{enumerate}

            \item U kombiju ima 8 sedi\v sta. Na koliko na\v cina mogu da sednu:

            2 putnika; 3 putnika; 4 putnika?

            

            Razmi\v slja\v s: Ako je jedan putnik on mo\v ze da sedne na svako od 8 sedi\v sta.

            Zna\v ci ima 8 mogu\' cnosti.



            Zato je dobro da ra\v cuna\v s: $2$ puta po $8$, tj. $8 + 8$ jer svaki putnik

            mo\v ze da sedne na svako sedi\v ste, $8$ puta po $2$, tj. $2+2+2+...+2$,

            jer svako sedi\v ste mo\v ze da primi, na primer ili Petra ili Jovana.

            $8 \cdot 2 = 8 + 8 = 16 \quad \text{i} \quad 2\cdot 8 = 2+2+2+2+2+2+2+2 = 16$

            Kako mo\v ze\v s zamisliti proizvod dva broja?



            Zami\v sljam \v semu:

            \begin{figure}

            \centering

            %\begin{table} % TODO

            %\end{table}

            \end{figure}



            Vidim da je to skup \v ciji su elementi raspore\dj eni u redove i kolone.

            \begin{figure}[h]

                \centering

\Placeholder[(10, 3)]{tackice 8 x 2}
            \end{figure}

            $16 = 8 \cdot 2$ [ $2$ reda po $8$ elemenata (sedi\v sta), mogu\' cnosti ]

            $16 = 8 \cdot 2$ [ $8$ kolona po $2$ elementa (putnika), mogu\' cnosti ]



            Tada se pi\v se broj $16$ i obliku proizvoda broja $2$ i broja $8$ i dobija se

            jednakost $$8 \cdot 2 = 2 \cdot 8$$



            \item Automobil ima $2$ sedi\v sta ($3$ sedi\v sta, $4$ sedi\v sta)

            i treba da sedne $8$ putnika. Na koliko na\v cina je to mogu\' ce?



            Ra\v cuna\v s: $8$ puta $2$, tj $2+2+...+2$, Jer svaki putnik mo\v ze da

            sedne na svako sedi\v ste, i $2$ puta po $8$, tj $8+8$, jer sedi\v ste mo\v ze

            da primi ili prvog, ili drugog, \dots, ili osmog putnika.

            $$2 \cdot 8 = 2+2+2+2+2+2+2+2 = 16, \text{ i } 8 \cdot 2 = 8 + 8 = 16$$



            To se vidi i iz slede\' ce \v seme:

            \begin{figure}

            \begin{tabular}{c c}

                %\begin{table}

                %\end{table}

\Placeholder[(2, 8)]{tackice 2 x 8}
                &

                %\begin{table}

                %\end{table}

\Placeholder[(2, 8)]{tackice 2 x 8}
            \end{tabular}

            \end{figure}

            Da je to skup \v ciji su elementi raspore\dj eni u redove i kolone (\ref{128:tabela1}).



            $16 = 2 \cdot 8$ [$8$ redova po $2$ elemenata, mogu\' cnosti]

            $16 = 8 \cdot 8$ [$2$ kolone po $8$ elemenata, mogu\' cnosti]



            Tako se pi\v se broj $16$ u obliku proizvoda broja $2$ i broja $8$ i dobija jednakost

            $$2 \cdot 8 = 8 \cdot 2$$

            Napi\v si ove jednakosti:

            $$8 \cdot 2 = 2 \cdot 8 \qquad 2 \cdot 8 = 8 \cdot 2$$

            $$16 = 8 \cdot 2 \qquad 2 \cdot 8 = 16$$

            $$16 = 2 \cdot 8 \qquad 8 \cdot 2 = 16$$



            Svaki proizvod dva broja ozna\v cava da te brojeve treba pomno\v ziti. Ka\v ze\v s broj $8$ treba 

            pomno\v ziti brojem dva, ili broj dva pomno\v ziti brojem osam, a pita\v s kratko:

            $2$ mno\v zi $8$ ili $8$ mno\v zi $2$.

        \end{enumerate}

    \end{zad}

    \begin{zad}

        Kupio sam $6$ svezaka. Svaka staje $6$ dinara. Ra\v cunaj.



        Platio sam u dinarima: $5 \times 6$ (6 mno\v zim 5)

        ovaj proizvod pi\v sem u obliku zbira:

        $$5 \times 6 = 5 + 5 + 5 + 5 + 5 + 5$$



        Zatim usmeno ra\v cuna\v s, pri \v cemu misli\v s:



        $2$ puta $5$ je $5$ i $5$, \dots $10$, $3$ puta $5$ je $10$ i $5$ je $15$, $4$ puta $5$ je $15$ i $5$ \dots $20$

        $5$ puta $5$ je $20$ i $5$ \dots $25$, $6$ puta $5$ je $25$ i $5$ \dots $30$

        $$5 \times 6 = 30$$



        To neka bude osnovni po\v cetak. Ako ne zna\v s $6$ puta $5$, izra\v cuna\v s $5$ puta $5$ i jo\v s $5$. Ako ni to ne

        zna\v s $3$ puta $5$ \dots $15$, $4$ puta $5$ \dots $20$, $5$ puta $5$ \dots $25$, $6$ puta $5$ \dots $25$ i $5$ \dots $30$.

    \end{zad}

    \begin{zad}

        Kupljeno je $8$ kutija u svakoj je $12$ olovaka.

    \end{zad}

    \begin{zad}

        Radnik je doneo $8$ puta po $10$ crepova.

    \end{zad}

    \begin{zad}

        Dragan ho\' ce da zasadi sadnice \v sljiva. Napravio je $7$ redova u svakom $12$ rupa. \v Sta ti mo\v ze\v s da

        izra\v cuna\v s? Izra\v cunaj.

    \end{zad}



    Ako zamisli\v s redove i iskopane rupe, to izgleda ovako:

    \begin{figure}

\Placeholder[(10, 4)]{prvi red ........ 12 rupa, drugi red .......... 12 rupa, ..., sedmi red .......... 12 rupa}
    \end{figure}

    $7$ redova po $12$ rupa je proizvod $12 \cdot 7$



    Dragan treba da iskopa rupa: $12 \cdot 7$

    Proizvod prikazujem u obliku zbira:



    $$12 \cdot 7 = 12 + 12 + 12 + 12 + 12 + 12 + 12$$



    Usmeno ra\v cunam tako \v sto po\v cinjem iz po\v cetka i govorim:

    \begin{itemize}

        \item $2$ puta $12$ je $12$ i $12$ \dots $24$

        \item $3$ puta $12$ je $24$ i $12$ \dots $36$

        \item $4$ puta $12$ je $36$ i $12$ \dots $48$

        \item $5$ puta $12$ je $48$ i $12$ \dots $60$

        \item $6$ puta $12$ je $60$ i $12$ \dots $72$

        \item $7$ puta $12$ je $72$ i $12$ \dots $84$

    \end{itemize}



    Dragan treba da iskopa ukupno rupa i zasadi ukupno sadnica: $12 \cdot 7 = 84$.

    \begin{zad}

        Zamisli $8$ linija nacrtanih cleva nadesno i $5$ linija nacrtanih tako da svaka od njih se\v ce svaku od

        ovih $8$ linija. Zamislim zrno pasulja. Na koliko na\v cina ga mo\v ze\v s staviti tako da ono le\v zi na

        dvema linijama? Za\v sta?

    \end{zad}

    \begin{figure}

        \centering

\Placeholder[(5, 5)]{3x3 linije}
    \end{figure}



    Dobio sam \v semu pomo\' cu kojke mogu izra\v cunati na koliko se na\v cina zrno pasulja mo\v ze postaviti,

    zato \v sto su preseci tih linija ta\v cke (mesta) na kojima treba staviti zrno pasulja. One su raspore\dj ene u redove i stupce (kolone), a to je proizvod brojeva $5$ i $8$.



    $8$ redova po $5$ polo\v zaja, to je proizvod $5 \cdot 8$ koji se zapisuje u obliku zbira:

    $$5 \cdot 8 = 5 + 5 + 5 + 5 + 5 + 5 + 5 + 5$$

    Ili $5$ stupca po $8$ polo\v zaja to je proizvod $8 \cdot 5$ koji se zapisuje u obliku zbira:

    $$8 \cdot 5 = 8 + 8 + 8 + 8 + 8$$

    Ra\v cunam od po\v cetka ,,redom''



    \begin{multicols}{2}

    \begin{align*}

        5 \cdot 1 &= 5 \\

        5 \cdot 2 &= 5 + 5 = 10\\

        5 \cdot 3 &= (5 + 5) + 5 = 10 + 5\\

        5 \cdot 4 &= (5 + 5 + 5) + 5 = 15 + 5 = 20\\

        5 \cdot 5 &= 20 + 5 = 25\\

        5 \cdot 6 &= 25 + 5 = 30\\

        5 \cdot 7 &= 30 + 5 = 35\\

        5 \cdot 8 &= 35 + 5 = 40

    \end{align*}

    \break

    \begin{align*}

        8 \cdot 1 = 8 \\

        8 \cdot 2 = 8 + 8 = 16 \\

        8 \cdot 3 = (8 + 8) + 8 = 24 \\

        8 \cdot 4 = 24 + 8 = 32 \\

        8 \cdot 5 = 32 + 8 = 40

    \end{align*}

    \end{multicols}

    Zrno pasulja se mo\v ze straviti na $40$ na\v cina.



    \begin{zad}

    U daljem radu ,,slu\v zi se i ovim specijalnim postupcima'' % TODO ref [1]

    \end{zad}

    \begin{enumerate}

        \item Kad zna\v s da $4$ je $2$ puta $2$, $6$ je $2$ puta $3$, $8$ je $2$ puta $4$, itd. ra\v cuna\v s:

        \begin{itemize}

            \item $2$ puta $10$ je $20$, a $4$ puta $10$ je $2$ puta $20$ \dots $40$

            \item $2$ puta $12$ je $24$, $4$ puta $12$ je $2$ puta $24$ \dots $48$

            \item $8$ puta $12$ je $2$ puta $48$ \dots $96$

            \item $2$ puta $15$ je $30$, $4$ puta $15$ je $2$ puta $30$ \dots $60$

            \item $2$ puta $3$ je $6$, $4$ puta $3$ je $2$ puta $6$ \dots $12$

            \item $2$ puta $5$ je $10$, $4$ puta $5$ je $2$ puta $10$ \dots $20$

            \item $2$ puta $9$ je $18$, $4$ puta $9$ je $2$ puta $18$ \dots $36$

            \item $8$ puta $9$ je $2$ puta $36$ \dots $72$

        \end{itemize}

        Zna\v ci, ako ne zna\v s koliko je $8$ puta $8$, ra\v cuna\v s $4$ puta $8$, odnosno $2$ puta $8$,

        pa onda $4$ puta $8$ i $8$ puta $8$ ($2$ puta $8$ \dots $16$, $4$ puta $8$ je $2$ puta $16$ \dots $32$,

        $8$ puta $8$ je $2$ puta $32$ \dots $64$).



        Mno\v zenje brojeva $7$, $8$ i $9$ treba da izra\v zunava\v s ovako:



        $5$ puta $8$ je $5$ puta $10$, mo\v ze $5$ puta $2$, dakle $50$ manje $10$;

        (simbolima kra\' ce: $8 \cdot 5 = (10 \cdot 2) \cdot 5 = 10 \cdot 5 - 2 \cdot 5 = 50 - 10 = 40$).



        $4$ puta $7$ je $4 \cdot 10$ manje $4$ puta $3$, dakle $40$ manje $12$.

        (simbolima kra\' ce: $7 \cdot 4 = (10 - 3) \cdot 4 = 10 \cdot 4 - 3 \cdot 4 = 40 - 12 = 28$)



        $9$ puta $8$ je $10$ puta $8$ manje $8$, dakle $80$ manje $8$;

        (kra\' ce $8 \cdot 9 = 8 \cdot (10 - 1) = 8 \cdot 10 - 8 \cdot 1 = 80 - 8 = 72$).



        $8$ puta $9$ je $8$ puta $10$ manje $8$, dakle, $80$ manje $8$;

        (kra\' ce: $9 \cdot 8 = (10 - 1) \cdot 8 = 10 \cdot 8 - 10 = 80 - 8 = 72$ ).



        Ovaj postupak je posebno podesan za mno\v zenje broja $9$ i mno\v zenje brojem $9$.

        \item $3$ puta $7$ je $21$, $6$ puta $7$ je $2$ puta $21$;

            (kra\' ce: $7 \cdot 3 = 21,\, 7 \cdot 6 = (7 \cdot 3) \cdot 2 = 21 \cdot 2 = 42$)

            

            $3$ puta $8$ je $24$, $6$ puta $8$ je $2$ puta $24$;

            ($8 \cdot 3 = 24,\, 8 \cdot 6 = (8 \cdot 3) \cdot 2 = 24 \cdot 2 = 48$)



            $3$ puta $9$ je $27$, $6$ puta $9$ je $2$ puta $27$;

            ($9 \cdot 3 = 27,\, 9 \cdot 6 = (9 \cdot 3) \cdot 2 = 27 \cdot 2 = 54$)

        \item $7$ puta $5$ je $4$ puta $5$ vi\v se $3$ puta $5$;

            (kra\' ce: $5 \cdot 7 = 5 \cdot (4 + 3) = 5 \cdot 4 + 5 \cdot 3 = 20 + 15 = 35$)



            $7$ puta $5$ je $5$ puta $5$ vi\v se $2$ puta $5$;

            (kra\' ce: $5 \cdot 7 = 5 \cdot (5 + 2) = 5 \cdot 5 + 5 \cdot 2 = 25 + 10 = 35$)



            $7$ puta $5$ je $10$ puta $5$ manje $3$ puta $5$;

            (kra\' ce: $5 \cdot 7 = 5 \cdot (10 - 3) = 5 \cdot 10 - 5 \cdot 3 = 50 - 15 = 35$)



            $7$ puta $8$ je $10$ puta $8$ manje $3$ puta $8$, dakle $80$ manje $24$;

            (kra\' ce: $8 \cdot 7 = 8(10 - 3) = 8 \cdot 10 - 8 \cdot 3 = 40 - 24 = 56$)



            $7$ puta $8$ je $5$ puta $8$ vi\v se $2$ puta $8$, dakle $40$ vi\v se $16$;

            (kra\' ce: $8 \cdot 7 = 8 (5 + 2) = 8 \cdot 5 + 8 \cdot 2 = 40 + 16 = 56$)

    \end{enumerate}

    

