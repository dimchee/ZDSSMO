    \section{Pismeno ra\v cunanje sabiranje}

    \begin{zad}

        Ako zna\v s postupak zasnivanja brojeva na pozicioni na\v cin, a to je bilo

        mogu\' ce uvo\dj enjem raznih sistema brojanja, onda \' ce ti biti lako da

        otkrije\v s postupak pismenog sabiranja. % todo (vidi posebno 364. zadatak).

        

        Slika \ref{r206:slika1} prikazuje skup u\v cenika jednog odeljenja,

        slika \ref{r206:slika2} prikazuje skup u\v cenika drugog odeljenja istog razreda.

        

        Izra\v cunaj broj u\v cenika tog razreda.

        

        \begin{figure}[H]

            \centering

\Placeholder[(8, 2)]{}
            \caption{}

             \label{r206:slika1}

        \end{figure}

        \begin{figure}[H]

            \centering

\Placeholder[(8, 2)]{}
            \caption{}

             \label{r206:slika2}

        \end{figure}



        Skup u\v cenika celog razreda prikazan na slici \ref{206:slika3} je

        grupisan po $3$, jer su dati skupovi tako grupisani.

        \begin{figure}[H]

            \centering

\Placeholder[(10, 2)]{}
            \caption{}

            \label{206:slika3}

        \end{figure}



        Mo\v ze se dogoditi da izra\v cuna\v s (\ref{r206:slika1}, \ref{206:slika2}) ovako:

        $$9 \cdot 5 + 3 \cdot 3 + 3 = 45 + 9 + 3 = 57$$.

        A mo\v ze\v s jednostavno da prebroji\v s sve elemente datih skupova.



        Mo\v ze\v s zbir da napi\v se\v s u sistemu 3, ra\v cunaju\' ci:

        $$18 \cdot 3 + 3 + 0 = 9 \cdot 2 \cdot 3 + 3 = [(3 \cdot 3) \cdot 3] \cdot 2 + (3 \cdot 3) \cdot 0 + 3 \cdot 1 + 0 = 2010_3$$



        Ovaj poslednji je dobro izra\v cunati po\v sto (slika \ref{206:slika3}) su

        elementi datih skupova grupisani po $3$, ali je to ipak

        ,,glomazan posao''.



        Zato sastaviti uniju skupova, njene elemente grupi\v si po tri, pa onda napi\v si

        odgovaraju\' ci broj (zbir) slika \ref{206:slika3}.



        Kooriste\' ci sliku % todo 184 zadatka 363

        i dobijenu uniju skupova (slika \ref{206:slika4})

        \begin{figure}[H]

            \centering

\Placeholder[(12, 2)]{}
            \caption{}

            \label{206:slika3}

        \end{figure}

        Dobijeni broj je $2010_3$.



        I ovde gubi\v s dosta vremena, zato napi\v si odgovaraju\' ce brojeve datih

        skupova kad je osnova brojanja tri. Probaj da sabere\v s ta dva broja onako kako

        su elementi unije grupisani.



        Broj provog skupa je $(3 \cdot 3) \cdot 2 + 3 \cdot 2 + 2 = 222_3$

        a broj drugog skupa je

        $(3 \cdot 3 \cdot 3) \cdot 1 + (33) \cdot 0 + 3 \cdot 1 + 1 = 1011_3$

        

        To je $222_3 + 1011_3 = 2010_3$.



        Prvo je od $1$ i $2$ jedinice sastavljena  $1$ trojka i $0$ jedinica tj.:

        

        $(1 + 2)$ jedinice je $3$ jedinice $= (3 + 0) = 1$ trojka i $0$ jedinica

        pi\v sem a $3$ trojke pamtim $(3 \cdot 3)$ tj. $1$ devetku.



        $(1 + 1 + 2)$ trojke = $4$ trojke $= (3 + 1)$ trojka, $1$ trojku pi\v sem,

        a $3$ trojke pamtim $(3 \cdot 3)$ tj. $1$ devetku.



        $(1 + 2)$ devetke $= 3$ devetke = $(3 + 0)$ devetke, $0$ devetke pi\v sem, a $3$ devetke pamtim

        $(9 \cdot 3) = (3 \cdot 3 \cdot 3) \cdot 1 = 1$ dvadeset sedmica



        $(1 + 1)$ dvadeset sedmica tj. $(3 \cdot 3 \cdot 3) \cdot 2$ i zato pi\v sem $2$ i ne pamtim ni\v sta

        (zavr\v savam sabiranje).



        Zato je $(3 \cdot 3 \cdot 3) \cdot 2 + (3 \cdot 3) \cdot 0 + 3 \cdot 1 + 0 = 2010_3$.



        Uo\v ci jednostavniji postupak:

        $(1 + 2)$ jedinice $= 3$ jedinice $= 0$ jedinice 

        \begin{center}

            \begin{tabular}{ p{1cm} p{1cm} }

                {\begin{align*}

                    \\

                    (1 + 2) \text{jedinice } = 3 \text{jedinice} \\ 

                    (1 + 1 + 2) \text{trojke} = 4 \text{trojke} (1 + 3)  \\

                    \\

                    (1 + 0 + 2) \text{ devetke } = 3 \text{devetke} \\

                    \\

                    (1 + 1) \text{ dvadeset sedmice } 

                \end{align*}}

                &

                {\begin{align*}

                    &\quad \text{pi\v sem} &\text{ i } \text{pamtim} \\

                    &= 0\text{ jedinice} &\text{ i } 1\text{ trojku} \\

                    &= 1\text{ trojku} &\text{ i } 3\text{ trojke} \\

                    &= 1\text{ trojku} &\text{ i } 1\text{ devetku} \\

                    &= (3 \cdot 3) \cdot 3 = 3 \cdot 3 \cdot 3 &\text{ i } (3 \cdot 3) \cdot 0 + (3 \cdot 3 \cdot 3) \cdot 1 \\

                    &= 0\text{ devetki} &\text{ i } 1\text{ dvadeset sedmica} \\

                    &= 2 \text{ dvadesetsedmice}

                \end{align*}}

            \end{tabular}

        \end{center}

        zbir je $2010_3$



        jo\v s preglednije:

        \begin{figure}[H]

            \centering

\Placeholder[(10, 5)]{$222_3 + 1011_3 = 2010_3$}
        \end{figure}



        Provera:

        \begin{align*}

        222_3 &= (3 \cdot 3) \cdot 2 + 3 \cdot 2 + 2 &= 18 + 6 + 2 &= 26_10 \\

        1011_3 &= (3 \cdot 3 \cdot 3) \cdot 1 + (3 \cdot 3) \cdot 0 + 3 \cdot 1 + 1 &= 27 + 0 + 3 + 1 &= 31_10 \\

        2010_3 &= (3 \cdot 3 \cdot 3) \cdot 2 + (3 \cdot 3) \cdot 0 + 3 \cdot 1 + 0 &= 54 + 0 + 3 + 1 &= 57_10

        \end{align*}

        Tj.

        \begin{align*}

            222_3 + 1011_3 &= 2010_3 \\

            26_10 + 31_10 &= 57_10

        \end{align*}

        je ta\v cno.

    \end{zad}

    \begin{zad}

        Izra\v cinaj: $314_5 + 404_5 = 1223$

        \begin{figure}[H]

            \centering

\Placeholder[(8, 5)]{$314_5 + 404_5 = 1223$}
        \end{figure}

    \end{zad}



    Provera 

    \begin{align*}

    314_5 &= (5 \cdot 5) \cdot 3 + 5 \cdot 1 + 4 &= 75 + 5 + 4 &= 84_{10} \\

    + 404_5 &= (5 \cdot 5) \cdot 4 + 4 \cdot 0 + 4 &= 100 + 0 + 4 &= 104_{10} \\

    \cline{1-4}

    1223_5 &= (5 \cdot 5 \cdot 5) \cdot 1 + (5 \cdot 5) \cdot 2 + 5 \cdot 2 + 3 &= 125 + 50 + 10 + 3 &= 188_{10} 

    \end{align*}

    

    Da li je ta\v cno $314_5 + 404_5 = 718_5$?



    Nije ta\v cno. Kada je osnova brojanja $5$ broj se ne mo\v ze zapisivati pomo\' cu cifara $5, 6, 7, 8, 9$

    

    Ali je ta\v cno $314_5 + 404_5 (=718) = 1223_5$.



    Zbir u zagradi je izvr\v sen u dekadnom sistemu. Zna\v s zbir nije zapisan u sistemu osnove $5$

    (cifre $5, 6, 7, 8, 9$ nisu cifre sistema osnove $5$).



    Zato \' ce\v s svaku cifru dekadnog sistema zapisati ciframa sistema osnove $5$.

    \begin{align*}

        718_{10} &= 1223_5 \\

        8_{10} &= 5_{10} + 3_{10} = 10_{5} + 3_{5} = 13_{5}, \text{ pi\v se\v s, a } 1 \text{ pamti\v s}. \\

        1 + 1 &= 2_{10} = 2_5, \quad 2 \text{ pi\v se\v s, ni\v sta ne pamti\v s}. \\

        7_{10} &= 5_{10} + 2_{10} = 10_5 + 2_5 = 12_5 \text{ pi\v se\v s } 12

    \end{align*}

    Izra\v cunaj primer iz prethodnog zadatka $222_3 + 1011_3$

    \begin{zad}

        Izra\v cunaj: $124_5 + 243_5; 645_7 + 563_7; 2201_3 + 2012_3; 10111_2 + 111_2; 3024_8 + 1532_8; 4716_{10} + 3987_{10};$

    \end{zad}

    $124_5 + 243_5 (= 367) = 422_5$, gde je $7 = 12_5$, $1 + 6 = 7 = 12_5$, $1 + 3 = 4_5$

    $645_7 + 563_7 (= (11)(10)8) = 1541_7$, gde je $8 = 11_7$, $1 + 10 = 14_7$, $1 + 11 = 12 = 15_7$



    Obrati pa\v znju ako sabiranjem cifara istog reda dobija\v s cifru dekadnog sistema, onda svaku cifru

    odgovaraju\' ceg reda pi\v se\v s u sistemu koji se tra\v zi.



    Ukoliko dobije\v s dvocifreni broj, pi\v se\v s ga u zagradi i njega izra\v zava\v s u sistemu koji se tra\v zi.



    $$3024_8 + 1532_8 (= 4556) = 4556_8$$



    Vidi\v s da je u ovom slu\v caju zbir cifara su cifre koje su cifre sistema u kome se vr\v si sabiranje.



    $$4716_{10} = 3987_{10} [= 7 (16) 9 (13)] = 8703_{10}$$



    \begin{zad}

        Izra\v cunaj $3421_5 + 1432_5$

    \end{zad}



    \begin{zad}

        Zbog sticanja potrebne brzine ui sabiranju vi\v se sabiraka, koristi\' ce\v s sabiranje

        pisanjem sabiraka jedan ishod drugog, umesto jedan iza drugog iz ,,prakti\v cnih'' razloga.



        Po\v cinje\v s od dva sabiraka, zatim tri i vi\v se sabiraka.



        Izra\v cunaj:  $587 + 654; 424_5 + 433_5; 645_7 + 536_7$



        \begin{tabular}{ p{3cm} p{3cm} p{3cm} }

        {\begin{align*}

            5&87& \\

         +  6&54& \\

         \cline{1-2}

            &11 &\text{ zbir jedinica}\\

            1&3 &\text{ zbir desetica}\\

            11& &\text{ zbir stotina}\\ 

        \cline{1-2}

            12&41&

        \end{align*}}

        &

        {\begin{align*}

            4&24_5 \\

         +  4&33_5 \\

         \cline{1-2}

            &12 \\

            1&0 \\

            13& \\ 

        \cline{1-2}

            14&12_5

        \end{align*}}

        &

        {\begin{align*}

            6&45_7 \\

         +  5&36_7 \\

         \cline{1-2}

            &14 \\

            1&0 \\

            14& \\ 

        \cline{1-2}

            15&14_7

        \end{align*}}

        \end{tabular}

    \end{zad}

