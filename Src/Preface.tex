\vspace*{\stretch{2}}
{\noindent \bfseries \Huge Predgovor}
\begin{center}
    \thispagestyle{plain}
\end{center}%

Poznato je da deca i mladi ljudi imaju jaku rođenu želju za samostalnim radom poznatom pod
izrekom ``hoću sam". To se tebi, korisniku ovog priručnika, pruža mogućnost da u svim postupcima
cam formiraš matematičke pojmove.

``Jer, iako je naporan, misaoni rad retko zamara. Zamara nerazumevanje i nerviranje zbog nerazumevanja".
%TODO cite[1]

Navikavanje na pravi misaoni rad uvek je moguće.

Zato je ova knjiga (priručnik) tako zamišljena, kao knjiga za osobu kojoj će uvek dobro doći za ulaženje u
matematiku, kao podsetnik i osnov daljeg usavršavanja.

Mnogi ljudi potcenjuju svoje sposobnosti i procenjuju tuđe. To nije dobro po samu ličnost koja se tako oseća.
U svakom slučaju budi uveren da za ulaženje u matematiku nije potrebna posebna pamet. Važno je da si uporna i
istrajna ličnost. Na ulaženje u matematiku utiču spoljni faktori, ipak je presudna ličnost osobe, tj. u prvom
redu njen odnos prema radu i njena upornost da postigne rezultate i onda kada joj posao ne ide kako bi želela.
Takva ličnost je svesna da je trnovit put do uspeha, tj. da je put do uspeha težak i naporan.

Ova kjiga će ti pomoći u ulaženju u matematiku, ali ona neće sama, treba ti to da želiš. To znači
treba da odlučiš da svaki zadatak samostalno rešiš po svaku, da pre nego što iscrpiš sve mogućnosti ne tražiš
tuđu pomoć ili gledaš uputstvo i odgovore.

Tako treba raditi sve dotle dok ne osetiš zadovoljstvo posle samostalno rešenog zadatka. Time si uspeo i niko
te ne može zaustaviti u rešavanju zadataka.

Na kraju treba da znaš da ma kako pristupaš određenom pojmu i ma šta preduzimaš, pojam se ne može formirati sve
dok ne uvidiš i ono što nije taj pojam, jer na primer ne postoji ''belo" bez pojma ``crno". Pojam
pravougaonika znaš kada znaš i šta nije pravougaonik (naprimer paralelogram). Podvlačim reč formiranje, jer
se pojmovi ne ``daju" ne ``uvode". Dakle, ``prvo se formira pojam, ma se onda prelazi na tehniku
savladavanja toga pojma" [1] %TODO cite
% TODO Spisak literature

\vspace*{\stretch{5}}
